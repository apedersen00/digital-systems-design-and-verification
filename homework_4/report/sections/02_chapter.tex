\documentclass[../main.tex]{subfiles}

\begin{document}

\section{Question 1} \label{sec:q1}

Calculation of the inverse of a 2 by 2 matrix is shown below.

$$
A =
\begin{bmatrix}
    a & b \\
    c & d
\end{bmatrix}
, \hspace{20pt} A^{-1} = \frac{1}{|A|}
\begin{bmatrix}
    d & -b \\
    -d & a
\end{bmatrix}
= \frac{1}{ad - bc}
\begin{bmatrix}
    d & -b \\
    -c & a
\end{bmatrix}
=
\begin{bmatrix}
    a_\text{out} & b_\text{out} \\
    c_\text{out} & d_\text{out}
\end{bmatrix}
$$

You are to design an RTL circuit for this calculation. When \texttt{start} is asserted, \texttt{aIn}, \texttt{bIn}, \texttt{cIn}, \texttt{dIn} 16-bit busses will contain the four elements of the matrix in upper left to lower right order. When the inverse calculation is completed, the IMC (Inverse Matrix Calculator) generates a 1 on \texttt{ready} and keeps this value until a new round of calculation begins. When calculation is completed, the output data becomes available on \texttt{aOut}, \texttt{bOut}, \texttt{cOut}, \texttt{dOut} output busses. Input and output data formats are 16-bit fixed point wiht eight integer bits. The inputs have only integer parts, and the outputs are 16-bit data with integer and fractional parts. The input integer part should be less than or equal to 15.

You can use the following components:

\begin{itemize}
    \item 2 16-bit unisgned multipliers with 16-bit inputs and a 16-bit outputs.
\end{itemize}

\end{document}
